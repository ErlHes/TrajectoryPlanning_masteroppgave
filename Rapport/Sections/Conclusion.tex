\blankpage
\section{Conclusion and Future Work}
% \begin{itemize}
%     \item conclusion:
%     \subitem oppsummering, forklaring, avsluttende ord.
%     \item future work:
%     \subitem (variabel) Cost funksjon
%     \subitem 'grenseverdier', altså verdier som constraint størrelse, distanse fra statiske hindringer, verdier som egentlig burde tunes basert på situasjonen slik den er i øyeblikket.
%     \subitem plassering av dynamiske constraints.
%     \subitem bedre måte å gjøre COLREGs assessment (ikke bare skjekk waypoints slik jeg gjør).
%     \subitem generelt andre metoder jeg ville foreslått å prøve isteden for spaghettien jeg har kokt sammen.
% \end{itemize}

\subsubsection*{Conclusion}
% \begin{itemize}
%     \item Summary of results, obviously.
%     \item Compare with the 'problem description', have I succesfully contributed to anything?
%     \item Final thoughts on my own work
% \end{itemize}


Using a combination of maneuvering theory and numerical optimal control, a trajectory planning and collision avoidance
algorithm has been developed. The algorithm was tested in two different configurations, one with perfect information about
other vessel's future trajectories, called "full prediction", and the other using linear interpolation to estimate other vessel's future trajectories, called "simple prediction".
In testing, it was found that having full prediction more often than not led to better COLREGs compliance in a variety of situations, all other factors equal.
With simple prediction the developed algorithm would often encounter hitches as the constraints on the optimal control problem would move between iterations,
which led to poorer performance both in terms of computational efficiency and observable COLREGs compliance.

While the technology for having full prediction does not yet exist, this thesis has shown that achieving better prediction than simple linear interpolation
would be beneficial even with no other changes made. Research into bettering the technology for intent inferring and trajectory predictions is therefore
a worthy pursuit.

Formulating the control objective as a NonLinear Programming problem, using Model Predictive Control (MPC) as a means for guiding the Own Ship has also 
shown itself to be a viable method for mid-level trajectory planning and collision avoidance. Using the CasADi framework and an IPOPT solver to solve the
optimal control problem, the algorithm was able to calculate the optimal trajectory for the next 5 minutes of transit in 0.7 to 3 seconds under normal conditions.
Under strained conditions or if the optimal control problem became infeasible it could take upwards of 90 seconds for the algorithm to arrive at a solution, though
nominal times under these tougher conditions were usually in the 10-30 seconds range.

The algorithm is able to navigate congested waters, avoid static obstacles, and slow down in the event of a blockade the Own Ship can't get past.
The algorithm features COLREGs situation assessment functionality, which is able to correctly identify which of the COLREGs 
rules apply to the Own Ship when encountering another vessel.

All that said, the algorithm suffers from a fair few shortcomings in terms of missing functionality or hard-coded parameter values that only work in a certain type
of situations. Among the aspects of the algorithm which are not satisfactory are: 
\begin{itemize}
    \item Inadequate situational awareness for threshold values that dictate safety.
    \item A static cost function, unable to adjust to COLREGs situations.
    \item Lack of adaptability when placing dynamic constraints.
    \item No functionality for culling out of reach or overlapping constraints.
    \item Hard-coded discretization step length, not possible to shorten for handling complicated situations, or extended when there is nothing happening.
    \item The dynamic horizon for the MPC could incorporate more variables than it currently does, most importantly it should be shortened in response to
    having a lot of active COLREGs situations.
\end{itemize}
These aspects are the basis of future improvement work.




\clearpage
\subsubsection*{Future Work}
% \begin{itemize}
%     \item More work needed for optimizing placement of constraint, as well as when constraints need to be active.
%     \item More work needed for situational awareness adjustion of parameters
%     \item more work needed to examine scenarios with and without full \gls{Ts} prediction;
%     more velocities, more variation of vessel sizes, more diverse enviornments.
%     \item More work related to a variable cost function, a more adaptable cost function could for example yield better COLREGs compliance.
%     \item More work related to Optimizing runtime of algorithm, tuning the IPOPT tolerances to balance computation speed and desired behaviour.
%     \item More work related to tuning COLREGs compliance, testing more COLREGs situations and quantifying what constitutes good behaviour.
%     \item Extracting the Algorithm from the MATLAB simulator it's built into and making a more stand-alone / generic software or algorithm.
%     \item And more that I will think about as I write.
%     \item dynamic discretization step length.
%     \item remove overlapping constraints.
% \end{itemize}

% Multiple avenues for improvements have already been suggested sporadically in the thesis. The most important ones are:
% \begin{itemize}
%     \item Dynamic cost function, being able to adjust the cost parameters in response to either COLREGs rules changing or unmodelled disturbances could
%     greatly improve the algorithms COLREGs compliance and general performance.
%     \item Dynamic thresholds for safety limits, this includes:
%     \subitem the dCPA and tCPA thresholds for making a COLREGs assessment, it should take into account vessel size and speed.
%     \subitem Lower bounds limit for static obstacles, shrinking the bound in low speed environments or if the optimal control problem is infeasible for too long.
%     \item Adaptable placement of dynamic constraints, to emulate true COLREGs compliance more logic is needed for the placement of dynamic constraints. The current
%     implementation of the algorithm does for example not take action early enough when in open waters, which could be solved by placing constraints further away from the Target Ship. But
%     the implementation needs to be sensitive to complex situations where available space is limited.
%     \item A reference filter for both the reference trajectory feeding into the trajectory planner, and for the output values from the algorithm, parsing the references
%     through a reference filter ensures feasibility at all times.
%     \item More input parameters for the dynamic horizon function, letting more factors such as amount of active COLREGs situations dictate how long the time horizon should be
%     to ensure that the algorithm is always able to solve the optimal control problem fast enough. This also includes functionality for having a dynamic discretization step length, which
%     could be shorter when the situation is complex and longer when there is not much going on.
% \end{itemize}
Multiple avenues for improvements have already been suggested sporadically in the thesis. The first one being a look at implementing a dynamic cost function.
The ability to adjust the cost parameters in response to either COLREGs rules changing or unmodelled disturbances could
greatly improve the algorithm's COLREGs-awareness and general performance. As the algorithm stands now, only constraints influence its COLREGs-awareness. This can be improved with a
cost function which penalizes turning and adjusting speed when in Stand-On, and encourages deviation from the reference trajectory when in Give-Way.

The next important area to improve is dynamic thresholds for safety limits, this includes values such as:
\begin{itemize}
    \item The Lower bounds limit for static obstacles.
    \item The dCPA and tCPA thresholds for making a COLREGs assessment.
    \item The size of circular dynamic constraints.
\end{itemize}
The two values for constraints are fairly self-explanatory, a static parameter for minimum distance to an obstacle will inevitably lead to problems. For example in a
port or a small canal it is okay for the minimum distance to obstacles to be small because the speeds are low. But on the ocean where big industrial ships are encountered
the minimum distance needs to be much bigger. The values for these thresholds need to take into account available information such as Target Ship's and Own Ship's sizes and velocities.
As for the tCPA and dCPA thresholds, these are used to determine if a Target Ship should be considered an active situation, making these thresholds dynamic based on the situation would improve
the COLREGs-awareness of the algorithm as well as performance as only active situation Target Ships need active constraints.

Another sought after feature would be adaptable placement of dynamic constraints. To emulate true COLREGs-awareness, more logic is needed for the placement of dynamic constraints. 
The current implementation of the algorithm does for example not take action early enough when in open waters, 
which could be solved by placing constraints further away from the Target Ship. But the implementation needs to be sensitive to complex situations where available space is limited.

Next, more input parameters for the dynamic horizon function. Including more factors such as amount of active COLREGs situations when deciding how long the time horizon should be
could improve the computational efficiency of the algorithm. When there are many active COLREGs situations the optimal control problem becomes more difficult to solve, shortening
the time horizon in these situations would ensure the algorithm is more capable at arriving at a solution quickly enough to take action. In a similar vein the discretization step
length should be dynamic so that the algorithm can have more or less control intervals depending on the complexity of the situation.

Over to some less important work, but still fairy valuable in terms of effort to reward ratio, would be having a reference filter on both reference trajectory and output from the algorithm.
Parsing the references and outputs through a reference filter would help ensure feasibility.



\newpage