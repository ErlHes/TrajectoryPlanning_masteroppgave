\newpage
\selectlanguage{Norsk}
\begin{abstract}
    \phantomsection
    \addcontentsline{toc}{section}{Sammendrag}
    Et fullt autonomt sjøfartøy må ha både baneplanlegging og kollisjonsungåelse systemer. Egenskapen til å følge et referansespor med tids- og andre
    effektivitetbegrensninger er helt essensielt for å ha et brukbart autonomt sjøfartøy. Det er heller ikke nok å bare følge en referanse, fartøyet må
    klare å unngå kollisjoner med andre fartøy som kan være både selvstyrt eller autonome i tillegg til statiske hindringer. Selv om det er et mål
    å unngå kollisjoner er det endelige målet med kollisjonsungåelse å klare å forholde seg til COLREGs reglene for sjøvett, regler som alle
    sjøfartøy må forholde seg til. I denne masteroppgaven skal det utvilkes en COLREGs forstående baneplanlegger algoritme som klarer å unngå 
    både statiske hindringer og andre fartøy. Algoritmen er basert på en kontrollmetode kalt MPC og bruken av numerisk optimalisering til baneplanlegging
    vil bli undersøkt som en del av oppgaven.

    Algoritmen will bli testet i forskjellige simulerte scenarier konstuert til å teste forskjellige deler av baneplanlegging og kollisjonsungåelse båe
    i individe- og hybridtester. Et annen punkt som vil bli undersøkt er bruken av en hypotetisk prediksjonsalgoritme som skal klare å forstå hensikten bak
    andre fartøys manøvere. Hvert scenarie kommer til å bli testet to ganger, den første gangen har den utviklede algoritmen tilgang til perfekt informasjon
    om andre fartøy, mens den andre gangen må algoritmen selv prøve å forutse banen til andre båter med linære metoder. Denne måten å teste på er ment for
    å undersøke potensielle fordeler med å utvikle mer avnaserte prediksjonsalgoritmer.
\end{abstract}

\afterpage{\blankpage}

