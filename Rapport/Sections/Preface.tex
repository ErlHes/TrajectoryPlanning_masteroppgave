\blankpage
\section*{Preface}
\addcontentsline{toc}{section}{Preface}
% Things to declare:\hfill\\
% MATLAB simulator files provided by Emil Thyri\\
% Guidance and general assistance: Morten Breivik, Emil Thyri\\
% ENC charts for scenario inspiration: Olex AS\\
% Casadi's Constrained Multiple Shooting example from example pack lay the foundation of the algorithm.\\
% \hfill\\
% Software used:\\
% MATLAB, Inkscape, Draw.io.\\
% \hfill\\
% Other tools:\\
% CasADi. IPOPT. MATLAB Mapping toolbox

This thesis was written as part of my M.Sc. degree in Industrial Cybernetics
at the Department of Engineering Cybernetics, Norwegian University of Science
and Technology (NTNU). The thesis is a continuation of my specialization project during the autumn of 2021.
I would like to thank my supervisors Morten Breivik and Emil Thyri, without whom I would have
never been able to fully understand and transform the subject matter into something worth writing about.\newline
Pivoting out of electrical engineering into the world of autonomous vessels has not been easy, for
the specialization project I took it as an absolute win to simply understand what I was doing. In this 
thesis, I was able to build upon that specialization and really experiment with the functionalities of the developed
algorithm.\newline
During the semester my supervisors have helped me with hour-long bi-weekly follow-up meetings, answered
all of my emails, and supplied me with research material and great sources. In the meetings, we discussed
not only progress, but concepts and practical ideas for improving my work.\newline
An extra thanks is extended to Emil Thyri for supplying me with a MATLAB simulator to use for testing the algorithm.
The algorithm was developed in MATLAB v2021b, using a framework by CasADi (v3.5.5) (\cite{andersson2019casadi}) and an IPOPT (\cite{wachter2006implementation}) solver. The algorithm
also uses the MATLAB mapping toolbox as well as MSS toolbox (\cite{MSStoolbox}). CasADi's example pack includes an example on 
direct multiple shooting which was used as a skeleton during development. Figures for the thesis were drawn
using Inkscape and Draw.io. Lastly I would like to thank Olex AS for letting me use their software to store AIS data
over the course of a few days to get a look at the ocean traffic along the Norwegian coast. This was used as inspiration
for the creation of simulated testing scenarios.
\begin{center}
    Erlend Hestvik\\
    Trondheim, 13.06.2022
\end{center}

\afterpage{\blankpage}