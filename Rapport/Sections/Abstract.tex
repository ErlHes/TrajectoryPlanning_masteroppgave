\newpage
\begin{abstract}
    \phantomsection
    \begin{itemize}
    A fully autonomous surface vessel will need both trajectory planning and collision avoidance systems. The ability to track a reference path
    with temporal or other efficiency constraints is essential for the usefulness of the \gls{ASV}. It is also not enough to simply follow a path
    optimally, the vessel must be able to evade collisions with other vessels both manned and unmanned as well as avoid static obstacles. While avoiding collisions
    is one goal, the real goal of collision avoidance is to achieve full \gls{COLREGs} compliance, a set of rules which every vessel on the sea must adhere to.
    In this thesis, a COLREGs aware trajectory planning algorithm capable of avoiding both static obstacles and other vessels is developed.
    The algorithm will be based on \gls{MPC} and the viability of numerical optimal control for mid-level trajectory planning will be examined.

    The algorithm will be tested in a variety of simulated scenarios designed to stress different aspects of trajectory planning and collision
    avoidance both individually and in combined situations. An additional point of examination will be the usage of hypothetical intention inferring methods
    and prediction of other vessels. Each scenario simulation will be conducted twice; in one version the developed algorithm is allowed near
    perfect information about the other vessels, while in the other it must linearly interpolate their course. This test will analyze the potential
    usefulness of improving prediction methods.
    \end{itemize}
\end{abstract}
\addcontentsline{toc}{section}{Abstract}
\afterpage{\blankpage}

