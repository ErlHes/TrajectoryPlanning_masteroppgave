\section{Introduction}
% \begin{itemize}
%     \item something something this thesis is about trajcetory planning
%     \item had this idea I wanted to try out
%     \item This chapter explains my motivation for the thesis. discusses previous work of the same subject.
%     Explains the problem as I see it, and my contributions to a solution. lastly an outline of the rest of the thesis for a quick intro of every section.
% \end{itemize}
% This chapter ...
\subsection{Motivation}
\begin{itemize}
    \item Worked on the same subject matter for a "fordypningsprosjekt" (finn godt ord).
    \item Autonomous vehicle control is an important milestone on the journey to a fully autonomous life.
    \item It's also just fricking cool on a conceptual level.
    \item Fishing industries and other marine industries are 'behind the curve' and not given as much attention as land based industries.
    \item A great learning opportunity for practical implementations of theory learned over the past two years.
    \item All in all a highly relevant project for the career trajectory I want.
    \item AI is pretty cool too I guess
    \item wanted to see if there could be a difference if autonomous vessels had more advanced prediction algorithms.
    \item just make something up.
    \item Find picture of some autonomous vessel or ferry
\end{itemize}

I remember when I took the subject TTK4115 - Linear System Theory; I thought to myself "I'm never gonna need this MPC stuff"... 
funny how that worked out.

\subsection{Previous Work}
\begin{itemize}
    \item cite \cite{loe2007collision} For in-depth look at many different methods.
    \item cite \cite{vagale2021path} For a review of path planning algorithms.
    \item cite \cite{zhang2021collision} For another big review on navigation systems for ASV
    \item cite \cite{huang2020ship} For another review of Collision avoidance.
    \item cite \cite{park2020social} For an alternative approach to Trajectory planning with similar-ish results.
    \item cite \cite{vestad2019automatic} for a nice study in route planning and sea lanes.
    \item Cite someone to prove that Autonomous surface vessels are real? :thinking\_emoji:
    \item item I *need* to use \gls{COLREGs} at least once so that the formating looks nice later.
\end{itemize}

\subsection{Problem Description}
% The problem that is addressed in this thesis is the trajectory planning for an \gls{ASV} operating in known environments where the motions are confined by
% a predefined area or path. The trajectory is to be planned in a way that ensures a collision-free transit from the current position to the destination. 
% The system needs to react to changes in the situation underway, and always deliver a collision-free trajectory. 
% It is assumed that the predefined area or path is free of any major static object, and the trajectory
% planning therefore only needs to consider the moving objects in the environment. The
% trajectory planner further needs to be encapsulated in a complete \gls{COLAV} system that
% interfaces with the situational awareness systems at one end, and the low-level motion
% control systems in the other end. The following objectives are proposed for this Master’s
% thesis
% \begin{itemize}
%     \item Develop a complete \gls{COLAV} system for an autonomous passenger ferry, with all
%     functionality that is needed for autonomous transits.
%     \item Implement interfaces to the situational awareness modules.
%     \item Implement interfaces to the trajectory following system.
%     \item Develop a simulator that can emulate moving objects in the environments.
%     \item Verify the performance of the \gls{COLAV} system through simulations.
%     \item Prepare the system for, and perform full-scale testing of the \gls{COLAV} system.
% \end{itemize}
The problem that are addressed in this thesis is the trajectory planning for an \gls{ASV} operating in calm waters
where the motions are confined by static obstacles such as land or skerries. The trajectory is to be collision
free at all times, and the behavior of the \gls{ASV} should comply with the COLREGs rules. The Trajectory is expected
to bring the \gls{ASV} from its current position to its end destination, but is not expected to be able to conduct
any type of docking, additionally the reference path between the \gls{ASV} current position and goal is predefined.
The system is to be able to operate under the assumption that has perfect information
about other vessels in the vicinity so that performance between linear interpolation of Target Ship trajectories
and full prediction of Target Ship trajectories can be tested.
The following objectives are proposed for this thesis:
\begin{itemize}
    \item Develop an MPC based trajectory planning algorithm using real vessel dynamics.
    \item Implement collision avoidance systems for both static and dynamic obstacles.
    \item Create functionality to ensure the vessel can not get stuck on terrain.
    \item Create test scenarios to examine the performance of the algorithm.
    \item Compare results when using linear interpolation of Target Ship trajectories and when using perfect information.
\end{itemize}


\subsection{Contributions}
\begin{itemize}
    \item A novel MPC based path following trajectory planner that accounts for both static and dynamic obstacles.
    \item An evaluation of the fitness of numerical optimization as trajectory planning backbone.
    \item Documented simulations experimenting with the difference 'Prediction Levels' make.
    \item Documented problems that numerical optimization based trajectory planner algorithms might encounter.
    \item Proposed mitigation methods for aforementioned problems.
\end{itemize}

\subsection{Outline}
\begin{itemize}
    \item Chapter 2 presents the full theoretical background needed to understand the development of the trajectory planning algorithm.
    \item Chapter 3 is a step-by-step walkthrough of the algorithm development.
    \item Chapter 4 compiles the results and analyzes some of the more interesting observations. The chapter is capped off with a greater
    discussion about the behavior of the algorithm, and it's shortcomings.
    \item Chapter 5 summarizes the findings and suggest a handful of avenues for future work that could improve the developed algorithm.
\end{itemize}


\newpage