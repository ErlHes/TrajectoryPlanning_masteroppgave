\blankpage
\section{Introduction}
% \begin{itemize}
%     \item something something this thesis is about trajcetory planning
%     \item had this idea I wanted to try out
%     \item This chapter explains my motivation for the thesis. discusses previous work of the same subject.
%     Explains the problem as I see it, and my contributions to a solution. lastly an outline of the rest of the thesis for a quick intro of every section.
% \end{itemize}
% This chapter ...
\subsection{Motivation}
% \begin{itemize}
%     \item Worked on the same subject matter for a "fordypningsprosjekt" (finn godt ord).
%     \item Autonomous vehicle control is an important milestone on the journey to a fully autonomous life.
%     \item It's also just fricking cool on a conceptual level.
%     \item Fishing industries and other marine industries are 'behind the curve' and not given as much attention as land based industries.
%     \item A great learning opportunity for practical implementations of theory learned over the past two years.
%     \item All in all a highly relevant project for the career trajectory I want.
%     \item AI is pretty cool too I guess
%     \item wanted to see if there could be a difference if autonomous vessels had more advanced prediction algorithms.
%     \item just make something up.
%     \item Find picture of some autonomous vessel or ferry
% \end{itemize}

In recent years automation has gotten a lot of media attention; self-driving cars, robots, drones. Everyone seems to be interested in
the possibilities that automation could afford us. Imagine no commute, no waiting for the delivery driver, robots to do our labor
for us. The journey towards a true post-scarcity society involves autonomous machines, and I want to participate in the process that speeds
up this development. Not only do I think automation could significantly improve the living and working conditions for everyone, I find
the concepts used when developing these automatas to be fascinating.

Autonomous surface vessels have mostly been in development outside the public eye, but there has been some great progress made in recent years. 
Multiple actors are currently developing both autonomous ferries and container ships, like Yara Birkeland seen in Figure \ref{FIG: Yara} 
or Zeabuz in Figure. 
This is a very underappreciated development, especially
in a coastal nation such as Norway where so much of our industry happens offshore.

\begin{figure}[h]
    \centering
    \includegraphics[width=0.8\textwidth]{Images/Yara.png}
    \caption{Yara Birkeland, Kongsberg Maritime's autonomous container ship project (\cite{KongsbergYara}). Image courtesy of Yara International ASA.}
    \label{FIG: Yara}
\end{figure}

This Master thesis is a continuation of my specialization project, (\cite{Hestvik2021}), in which I learned a lot about the complications
of designing a trajectory planning algorithm, numerical optimal control, and using the CasADi framework. After the specialization project
ended in December 2021, I couldn't leave the trajectory planning algorithm to be in the state it was, there was still so much more to try and learn.

The learning potential from this thesis is very high, and very relevant for the type of career I would like to get into.
I remember when learning about linear system theory I thought to myself "I'm never gonna need this MPC stuff". Funny how that worked out.

\begin{figure}
    \centering
    \includegraphics[width=0.8\textwidth]{Images/Zeabuz.png}
    \caption{Render of the Zeabuz autonomous urban ferry, currently in development. Image courtesy of \cite{Zeabuz}}
\end{figure}

\subsection{Previous Work}
% \begin{itemize}
%     \item cite \cite{loe2008collision} For in-depth look at many different methods.
%     \item cite \cite{vagale2021path} For a review of path planning algorithms.
%     \item cite \cite{zhang2021collision} For another big review on navigation systems for ASV
%     \item cite \cite{huang2020ship} For another review of Collision avoidance.
%     \item cite \cite{park2020social} For an alternative approach to Trajectory planning with similar-ish results.
%     \item cite \cite{vestad2019automatic} for a nice study in route planning and sea lanes.
%     \item Cite someone to prove that Autonomous surface vessels are real? :thinking\_emoji:
%     \item item I *need* to use \gls{COLREGs} at least once so that the formating looks nice later.
% \end{itemize}

There has been many reviews and surveys on the state of the art within \gls{ASV} trajectory planning and collision avoidance, and other
related topics. In their review, (\cite{vagale2021path}) examines existing guidance and path planning algorithms, comparing multiple aspects and highlighting
recent advances and vessel autonomy levels. The review focuses mostly on path planning algorithms for surface vessels.

In another review (\cite{huang2020ship}) provides a comprehensive overview of collision avoidance, breaking down the basic processes
for determining the need of an evasive maneuver. The review compares multiple methods and points out strengths and weaknesses of each.
The also examines the differences in research for manned and unmanned vessels, pointing out what the two groups can learn from each other.

A state of the art survey by (\cite{zhang2021collision}) examines some recent high-profile \gls{ASV}s, their purpose and development status. Later
the survey looks at different methods for collision avoidance, the tools different methods are utilizing and trend of where collision
avoidance systems appear to be heading.

For more specific recent development, (\cite{park2020social}) developed a trajectory planning algorithm for an ASV operating in urban canals. The
authors of this paper aim to emulate human like behavior in what they call social trajectory planning. Where the optimal control formulation
penalizes deviation of movements from nominal movements by human operated vessels. The developed method has no direct collision avoidance component,
instead focusing on the benefits of social trajectory planning to reduce the amount of encounters the ASV has during transit.

For longer distance path planning, (\cite{vestad2019automatic}) develops an automatic route planner which is able to quickly
determine an optimal path between two points taking into account ship dimensions, weather forecast and mission goals. The
path planner is able to produce feasible paths which are more efficient than traditional manual planning methods.

For a comprehensive look at different trajectory planning methods, (\cite{loe2008collision}) introduces many
approaches and simulates their performance against each other in a wide variety of scenarios. Testing both collision avoidance capabilities and
trajectory planning.

\subsection{Problem Description}
The problem that is addressed in this thesis is the trajectory planning for an \gls{ASV} operating in calm waters,
where the available maneuvering space is confined by static obstacles such as land and by dynamic obstacles such as other vessels.
The planned trajectory will continuously be replanned to adapt to changes in the situation and maintain COLREGs compliance.
The Trajectory is expected to bring the \gls{ASV} from its current position to its end destination without collisions, but is not expected to be able to conduct
any type of docking. Additionally, the reference path between the \gls{ASV} current position and goal is predefined.
The system is to be able to operate under the assumption that has perfect information
about other vessels in the vicinity so that performance between linear interpolation of Target Ship trajectories
and full prediction of Target Ship trajectories can be tested.
The following objectives are proposed for this thesis:
\begin{itemize}
    \item Develop an MPC-based trajectory planning algorithm using sufficiently real vessel dynamics.
    \item Implement collision avoidance algorithms for both static and dynamic obstacles.
    \item Create functionality to ensure the vessel can not get stuck on terrain.
    \item Create test scenarios to examine the performance of the algorithm.
    \item Compare results when using linear interpolation of Target Ship trajectories and when using perfect information in the planner.
\end{itemize}


\subsection{Contributions}
This thesis provides the following contributions to the field of \gls{ASV}:
\begin{itemize}
    \item An MPC based trajectory planner that is COLREGs-aware and able to avoid static obstacles.
    \item An evaluation of the fitness of numerical optimization as trajectory planning backbone.
    \item Designed and simulated testing scenarios for evaluating the developed algorithm.
    \item Documented simulations experimenting with two different levels of \gls{Ts} trajectory prediction capabilities. Where one prediction level
    gives the algorithm perfect information about the \gls{Ts}, while the other only provides a linearly interpolated trajectory based on the \gls{Ts} current pose.
    \item Documented problems that numerical optimization-based trajectory planner algorithms might encounter.
    \item Proposed mitigation methods for aforementioned problems.
\end{itemize}

\subsection{Outline}
The thesis is laid out in five parts, the introduction you just read was Chapter 1. The rest of the chapters are as follows:
\begin{itemize}
    \item Chapter 2 presents the full theoretical background needed to understand the development of the trajectory planning algorithm.
    \item Chapter 3 is a step-by-step walkthrough of the algorithm development.
    \item Chapter 4 compiles the results and analyzes some of the more interesting observations. The chapter is capped off with a greater
    discussion about the behavior of the algorithm and it's shortcomings.
    \item Chapter 5 summarizes the findings and suggests a handful of avenues for future work that could improve the developed algorithm.
\end{itemize}


\newpage