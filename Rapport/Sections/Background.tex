\section{Background}
\begin{itemize}
    \item Skal prøve å skrive med litt bedre rød tråd denne gangen
\end{itemize}
Before diving into the main matter of this thesis, some groundwork is neccessary to explain the fundamental theories that the thesis is built upon.
This section will provide a structured overview of the concepts and terminology that will be used later.

\subsection{Vessel Model}
\begin{itemize}
    \item Kinetikk.
    \item Kinematikk.
    \item hvorfor denne modellen.
    \item Hvordan brukes modellen. 
\end{itemize}

In this thesis, the terms \textit{vessel}, \textit{ship}, and \textit{boat} will all be used interchangably to refer to the same thing: a hollow structure that floats
on water for the purpose of navigation. Exactly how the vessel behaves as it floats depends on a multitude of factors which can be modelled mathematically. There are many
ways to model a vessel, but not every model is useful for every purpose. Just like how not every tool is fit as a hammer. To create a useful model for
navigation and control specifically, we separate the modelling problem into two categories: Kinematics and Kinetics.

\subsubsection{Kinematics}
Kinematics is the aspect of the problem concerned only with the geometrical aspect of motion. When modelling our hollow structure is not enough to simply 
describe the velocity which with it floats. If the vessel has it's own actuator to generate thrust we must know which direction the force generated is pointing, and
if there are external forces acting upon the vessel we must know which direction those forces are pointing. But what is the reference point for these directions? and where
even is our vessel floating?   ...\textit{Dette er ikke hvordan jeg ønsker å skrive dette avsnittet, men jeg lar det stå slik som det for nå...}

What we need is a reference frame, in fact we need multiple reference frames. A reference frame is simply a origin point and some defined direction or axis that together
come to define a coordinate system. One such frame is called Earth Centered Inetrial (ECI), this is an inertial frame where the z-axis is aligned with the axis which Earth
rotates around while the x and y axis are fixed in space. An observer using this reference frame would see the Earth spinning around the z-axis, the solar system would appear
to rotate around Earth, and objects which would be considered stationary on Earth's surface would have an angular velocity. Another useful frame is called
Earth Centered Earth Fixed (ECEF) and is very similar to ECI except that the x and y axis now rotate with Earth's rotation. Objects which are considered to be stationary on
the surface of Earth are also considered to be stationary to an observer using ECEF. EFEC is the frame in which Longitude, Latitude and Altitude are used to describe
a position and heigh, and this frame is widely used in satellite based global navigation systems such as Global Positioning System (GPS).
\\
There are two more reference frames which are very useful for modelling our vessel:
\begin{itemize}
    \item \textbf{NED} or North-East-Down is a flat plane projection of some area on Earth. The origin point for a NED frame is generally some Longitude-Latitude
    position from ECEF with the x-axis poiting towards true North, y-axis pointing east and the z-axis ponting down towards the center of the Earth. NED will be denoted as
    {n} from here on.
    \item \textbf{BODY} is a frame affixed to the body of the vessel with the x-axis aligned with the longitudal axis of the vessel from aft to fore, 
    the y-axis is the transversal axis pointing towards starboard, and the z-axis is the normal axis from top to bottom. BODY will be denoted as {b} from here on.
\end{itemize}

The reason for all these frames is that motion described by one frame can be transformed to be expressed in another, meaning that we can define the forces of our
vessels actuators in {b} and transform this to motion in {n}. Consider that the z-axis for {b} and {n} are parallel, rotating from {b} to {n} becomes
a principal rotation about the z-axis:
\begin{equation}
    \label{eq:principal rotation}
    J_{\Theta}(\eta) = R_{z,\psi} = \begin{bmatrix}
        cos(\psi) & -sin(\psi) & 0\\
        sin(\psi) & cos(\psi) & 0\\
        0 & 0  & 1\\
    \end{bmatrix} 
\end{equation}

This rotation is the relationship between body velocities and positional change:

\begin{equation}
    \label{eq:eta dot}
    \dot{\eta} = R(\psi) \nu
\end{equation}

where $R(\psi) := R_{z,\psi}$, $\nu = [u,v,r]'$ are the surge, sway and heave velocities of our vessel. And $\eta = [N, E, \psi]'$ is the north east position and heading orientation of our vessel.

\subsubsection{Kinetics}
\begin{itemize}
    \item Kinetics is the model for turning forces into motion.
    \item Degrees of Freedom explained
    \item 3DOF rigid body kinetics from Fossen 2011
    \item Explain added mass, coriolis, dampening.
    \item explain thrust?
\end{itemize}
Kinetics is the aspect of the modelling problem that is concerned with converting forces to motion.
something something 3DOF. something something mass, corriolis, dampening, thrust matrices.



\subsection{Collision Avoidance}
\begin{itemize}
    \item liten introsnutt om hva jeg mener går under paraplyen "Collision avoidance".
    \item Denne blir vel egentlig ganske lik som i fordypningsoppgaven.
\end{itemize}

\subsubsection{COLREGs}
\begin{itemize}
    \item COLREGs.
\end{itemize}

\subsubsection{Situation Assessment}
\begin{itemize}
    \item Samme som i fordypningsoppgaven.
\end{itemize}

\subsubsection{Target Ship Prediction / Situation Anticipation}
\begin{itemize}
    \item Vanlig metode (tCPA, dCPA).
    \item Avansert metode (hypotetisk, maskin lærings problem?).
    \item henger sammen med Situation assessment.
    \item Her kommer delen om trafikk pattern data for prediksjon.
\end{itemize}

\subsection{Trajectory planning algorithm}
\begin{itemize}
    \item Litt tekst om hvordan modellen og collision avoidance knyttes inn i trajectory planning.
\end{itemize}

\subsubsection{Line of Sight Guidance}
\begin{itemize}
    \item Hva er det.
    \item Hvordan gjøres det.
    \item Hva skal det brukes til.
\end{itemize}

\subsubsection{Optimal Control problem}
\begin{itemize}
    \item matematikk.
    \item Hvordan brukes alt vi har skrevet om tidligere i dette kapittelet.
    \item Hva får vi ut som svar.
\end{itemize}

\subsubsection{Model Predictive Control}
\begin{itemize}
    \item MPC kommer til å nevnes en god del, bør ha sitt eget lille kapittel.
    \item hva er det.
    \item Hvordan funker det.
    \item Hvorfor bruke MPC.
\end{itemize}

\subsection{Robot Operating System}
\begin{itemize}
    \item Skrive litt om ROS?
    \item importering av MATLAB kode.
    \item Dette delkapittelet kan muligens gå under metode.
\end{itemize}

\newpage