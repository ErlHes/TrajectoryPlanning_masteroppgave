\section{Background}
\begin{itemize}
    \item Husk rød tråd.
    \item Vær generisk.
    \item Bare inkluder konsept som blir relevante senere, eller som er brukt i nødvendige antagelser.
\end{itemize}

\subsection{Target Ship prediction}
\begin{itemize}
    \item Gjenfortelling fra fordypningsprosjekt, da kalt traffic pattern
    \item Fant en annen artikkel fra Kina som skrev om nogenlunde det samme, AIS data -> prediksjon
    \item Skiller seg fra fordypningsprosjekt fordi det er egentlig ikke traffic pattern som er den viktige antagelsen,
    Det er heller viktig at vi antar det finnes en måte å gjette/vite hvor andre båter vil være fremover i tid.
    \item Andre metoder for target ship prediction kan være f.eks utvidelse av AIS som inkluderer autonav data for de neste 5 minuttene eller noe lignende.
\end{itemize}

\subsection{ASV modelling}
Jeg tenker det er best å skrive om modellering i sammenheng med hvordan trajectory planning problemet blir satt opp i MATLAB med CasADi.
\begin{itemize}
    \item Kinematics \& Kinetics -> Begge brukes i CasADi setup
    %\item Munk moment -> Potensielt relevant hvis jeg skriver om modellerings problemer
    \item Her kan det også skrives om de spesifike tallverdiene som blir brukt i Masse, coriolis og dempnings -matrisene.
    de er spesifike til Milliampere, funnet gjennom en rekke forsøk utfort av Anders Pedersen.
\end{itemize}

\subsection{Trajectory Planning}
\begin{itemize}
    \item How to get from A to B.
    \item Multiple methods, all with pros and cons, skriv liten oversikt.
    \subitem LOS, OCP, Machine Learning, osv.
    \subitem Kanskje ikke så veldig viktig å snakke om andre metoder enn OCP.  
    \item Important factors to consider:
    \subitem Time horizon / length of planning period.
    \subitem Trajectory safety with respect to ship capabilities.
    \subitem COLREGs compliance with respect to expected behaviour.
    \subitem osv.
    \item Litt dypere inn i numerisk optimalisering og MPC, og LOS ettersom det kommer til å bli brukt igjen senere.
   % \item Først oversikt, deretter enten nytt kapittel eller delkapittel om 
   % den spesifikke metoden jeg skal bruke, forklar numerisk optimalisering, MPC, LOS guidance, hvordan alt kombineres til å bli min algoritme, in theory.
\end{itemize}


\subsection{Collision Avoidance}
\begin{itemize}
    \item COLREGs
    \subitem Expected behaviour, situation classification, etc etc.
    \item dCPA / tCPA
    \item Other risk assessment? Situation complexity? Det er mer som inngår i "collisions avoidance" som jeg kanskje ikke dekker så veldig bra med min algoritme.
\end{itemize}

\subsection{'The complete system'}
\begin{itemize}
    \item Vet ikke helt om dette kapittelet er nødvendig, men jeg lurer på om det er en god ide å skrive litt om nøyaktig hvor i ett fult funksjonelt
    system jeg forventer at min algoritme passer inn. Hva de andre delene jeg ikke kommer til å skrive om har ansvar for, og hva som forventes av systemene
    rundt mitt eget.
    \item Hvis systemet mitt var en sort boks, hvilke inputs og outputs ville det hatt.
\end{itemize}



\newpage