\blankpage
\section{Helper Functions}
\begin{lstlisting}
    function F = CasadiSetup(h, N)
    import casadi.*
    
    T = h * N;
    
        %% CasADi setup
        
        % System matrices.
        x = SX.sym('x',6); % x = [N, E, psi, u, v, r]'
        tau = SX.sym('tau',3); % tau = [Fx, Fy, Fn]';
        xref = SX.sym('xref',6); % xref = [Nref, Eref, Psi_ref, Surge_ref, sway_ref, r_ref]'
        
        
    %     [R, M, C, D] = SystemDynamics(x, u); % Usikker på hvorvidt det funker
    %     å sende CasADi systemer inn i en subfunksjon. Burde jo gå, men lar
    %     være for nå.
        % Model Parameters.
        Xu = -68.676;   % Kg/s
        Xuu = -50.08;   % Kg/m
        Xuuu = -14.93;  % Kgs/(m^2)
    %     Xv = -25.20;    % Kg/s
    %     Xr = -145.3;    % Kgm/s
    %     Yu = 90.15;     % Kg/s
        Yv = -8.69;     % Kg/s
        Yvv = -189.08;  % Kg/m
        Yvvv = -0.00613;% Kgs/(s^2) ? Kgs/(m^2)?
    %     Yrv = -3086.95; % Kg
    %     Yr = -24.09;    % Kgm/s
    %     Yvr = -338.32;  % Kg
    %     Yrr = 1372.06;  % Kg(m^2)
    %     Nu = -38.00;    % Kgm/s
    %     Nv = -97.26;    % Kgm/s
        Nvv = -18.85;   % Kg
        Nrv = 5552.23;  % Kgm
        Nr = -230.19;   % Kg(m^2)/s
        Nrr = -0.0063;  % Kg(m^2)
        Nrrr = -0.00067;% Kgms
    %     Nvr = -5888.89; % Kgm
        
        m11 = 2131.80;  % Kg
        m12 = 1.00;     % Kg
        m13 = 141.02;   % Kgm
        m21 = -15.87;   % Kg
        m22 = 2231.89;  % Kg
        m23 = -1244.35; % Kgm
        m31 = -423.76;  % Kgm
        m32 = -397.64;  % Kgm
        m33 = 4351.56;  % Kg(m^2)
        
        c13 = -m22*x(5);
        c23 = m11*x(4);
        c31 = -c13;
        c32 = -c23*x(5);
        
        d11 = -Xu - Xuu * abs(x(4)) - Xuuu*(x(4)^2);
        d22 = -Yv - Yvv*abs(x(5)) - Yvvv*(x(5)^2);
        d23 = d22;
        d32 = -Nvv*abs(x(5)) - Nrv *abs(x(6));
        d33 = -Nr - Nrr*abs(x(6)) - Nrrr*(x(6)^2);
        
        
        % System dynamics.
        R = [cos(x(3))    -sin(x(3))    0;...
             sin(x(3))    cos(x(3))     0;...
                0           0       1];    
        M = [m11   m12     m13;...
             m21   m22     m23;...
             m31   m32     m33];
        C = [0    0   c13;...
             0    0   c23;...
             c31 c32    0];
    %     D = [d11     0        0;...
    %          0      d22     d23;...
    %          0      d32     50*d33];
    % 
    %      M = eye(3)*1000;
         D = diag([200, 200, 1000]);
    %      C = zeros(3);
        
    %     Tau = pickthree(tau); %failed experiement.
        nu_dot = M\(tau -(C+D)*x(4:6)); 
        nu = x(4:6) + h*nu_dot; % This could almost certainly use a better integrator method.
        eta_dot = R*nu;
        
        xdot = [eta_dot; nu_dot];
        
    %     Funker bra:
    %     Kp = diag([8*10^-1, 8*10^-1]);
    %     Ku = 6*10^2;
    %     Kv = 8*10^2;
        
        % Objective function.
        Kp = diag([8*10^-1, 8*10^-1]); % Tuning parameter for positional reference deviation.
        Ku = 6.7*10^2; % Tuning parameter for surge reference deviation.
        Kv = 7.2*10^2;
    %     Kv = 0;
    %     Kr = 3*10^2; % Tuning parameter for yaw rate reference deviation.
    %     Kt = 10^2;
        R2 = [cos(x(3))    -sin(x(3));...
              sin(x(3))    cos(x(3))];
        Error = R2'*(x(1:2) - xref(1:2));
        Kfy = 1 * 10^-5;
    
        %Test for heading
        K_phi = 6*10^-5;
    
        %L = Kp * norm(P - xref)^2 + Ku  * (u(1) - uref(1))^2 + Kr * (u(2) - uref(2))^2;
        %L = (P - xref)'* Kp * (P - xref) + Ku * (u_0'*u_0 - uref(1)'*uref(1))^2;
        %L = (P - xref)'* Kp * (P - xref) + Ku * (u(1) - uref(1))^2 + Kr * (u(2) - uref(2))^2;
        L = Error'* Kp * Error + Ku * (x(4)-xref(4))^2 + Kv * (x(5)-xref(5))^2 + Kfy * tau(2)^2 + K_phi * (ssa(x(3)-xref(3)))^2;% + Kr * (x(6) - xref(6))^2 + Kt * (tau'*tau) + Ku * (x(4) - xref(4))^2;
        
        % Continous time dynamics.
        f = Function('f', {x, tau, xref}, {xdot, L});
        
        % Discrete time dynamics.
        M = 4; %RK4 steps per interval
        DT = T/N/M;
        f = Function('f', {x, tau, xref}, {xdot, L});
        X0 = MX.sym('X0',6);
        Tau = MX.sym('Tau',3);
        Xd = MX.sym('Xd',6);
        X = X0;
        Q = 0;
        for j=1:M
            [k1, k1_q] = f(X, Tau, Xd);
            [k2, k2_q] = f(X + DT/2 * k1, Tau, Xd);
            [k3, k3_q] = f(X + DT/2 * k2, Tau, Xd);
            [k4, k4_q] = f(X + DT * k3, Tau, Xd);
            X=X+DT/6*(k1 +2*k2 +2*k3 +k4);
            Q = Q + DT/6*(k1_q + 2*k2_q + 2*k3_q + k4_q);
        end
        
        F = Function('F', {X0, Tau, Xd}, {X, Q}, {'x0', 'tau', 'Xd'}, {'xf', 'qf'});
    end
        
\end{lstlisting}

\clearpage
\begin{lstlisting}
    function [dCPA, tCPA] = ClosestApproach(pos_OS, pos_TS, vel_OS, vel_TS)
%Returns the distance at closest point of approach and time until closest
%point of approach. Assuming both vessels maintain a fixed course and
%speed.
vel_AB = vel_OS - vel_TS;
pos_BA = pos_TS - pos_OS;

tCPA = 0;
if (norm(vel_AB,2) > 0)
    tCPA = dot(pos_BA,vel_AB) / norm(vel_AB,2)^2;
end

dCPAfunc = (pos_OS + tCPA * vel_OS) - (pos_TS + tCPA * vel_TS);
dCPA = norm(dCPAfunc,2);

if tCPA < 0
    dCPA = norm(pos_BA,2);
    tCPA = 0;
end

end
\end{lstlisting}

\clearpage
\begin{lstlisting}
    function [flag, dCPA, tCPA] = COLREGs_assessment(vessel,tracks, cflag)
    %% THIS FUNCTION EVALUATES ONE TARGET SHIP ONLY. TO EVALUATE MORE THE FUNCTION MUST BE CALLED FOR EACH TARGET SHIP IN YOUR SITUATION.
    % a13 = 112.5; % Overtaking tolerance
    % a14 = rad2deg(pi/8); % head-on tolerance
    % a15 = rad2deg(pi/8); % crossing aspect limit
    
    %% Calculate dCPA and tCPA, check if COLREGs assessment is needed:
    % [dCPA, tCPA] = ClosestApproach(vessel.eta(1:2), tracks.eta(1:2), vessel.eta_dot(1:2), tracks.eta_dot(1:2));
    [dCPAlist, tCPAlist,pos_OS_list, pos_TS_list] = getCPAlist(vessel,tracks);
    
    %Keep the lowest dCPA found, this is the only dCPA we're interested in
    %If there should ever be multiple equally low dCPAs we are in a unsupported
    %special case that needs more development.
    dCPA = min(dCPAlist);
    dCPAminlist = find(dCPAlist == dCPA);
    tCPA = tCPAlist(dCPAminlist(1));
    pos_OS = pos_OS_list(1:3,dCPAminlist);
    pos_TS = pos_TS_list(1:3,dCPAminlist);
    
    [TSdCPAlist, TStCPAlist, ts_pos_TS_list, ts_pos_OS_list] = getCPAlist(tracks,vessel);
    TSdCPA = min(TSdCPAlist);
    TSdCPAminlist = find(TSdCPAlist == TSdCPA);
    
    %HACKJOB
    %This is a failsafe to prevent MATLAB from throwing an error and halting
    %the program should any of the Target Ships in the simulation be at their
    %final destination.
    if(~isempty(TStCPAlist))
        TStCPA = TStCPAlist(TSdCPAminlist(1));
        tspos_OS = ts_pos_OS_list(1:3,TSdCPAminlist);
        tspos_TS = ts_pos_TS_list(1:3,TSdCPAminlist);
    else
        TStCPA = 0;
        tspos_OS = [0 0 0]';
        tspos_TS = [100 100]';
    end
    %END of HACKJOB
    
    if TSdCPA < dCPA
        dCPA = TSdCPA;
        tCPA = TStCPA;
        pos_OS = tspos_OS;
        pos_TS = tspos_TS;
    end
    
    %Nå vet vi hva dCPA og tCPA er, kan nå sammenligne med en eller annen
    %kvantitet for å se om det er høvelig å sette COLREGs flag på TS.
    %HVIS vi ønsker å sette COLREGs flag må vi også vite hvor OS og TS er i
    %forhold til hverandre, og hvilke kurs begge har når vi starter på banen
    %som tar oss til denne dCPAen.
    OSareal = vessel.size(1)*vessel.size(2);
    TSareal = tracks.size(1)*tracks.size(2);
    
    dCPAgrense = (OSareal + TSareal + max(OSareal,TSareal)) / 2; % En eller annen funksjon av størrelser
    %Hvis problemet blir unfeasible kan det hende vi blir nødt til å senke
    %denne grensen, men det er en funksjon for en annen dag.
    
    tCPAgrense = 3 * dCPAgrense;
    
    
    
    %% Conduct COLREGs assessment
    if (dCPA < dCPAgrense) && (tCPA < tCPAgrense) && cflag == 0
        % Angles between OS and TS
        phi_1 = rad2deg(pi/8);
    %     phi_1 = rad2deg(pi/15);
        phi_2 = 112.5;
        
        b0 = rad2deg(wrapTo2Pi(atan2((pos_TS(2)-pos_OS(2)),(pos_TS(1)-pos_OS(1))) - wrapToPi(pos_OS(3)))); % Relative from OS to TS
    
        b0_180 = rad2deg(wrapToPi(deg2rad(b0)));
        
        a0 = rad2deg(ssa(atan2(pos_OS(2)-pos_TS(2),pos_OS(1)-pos_TS(1)) - pos_TS(3))); % Relative from TS to OS
        
        % dist = sqrt((tracks.eta(2) - vessel.eta(2))^2 + (tracks.eta(1) - vessel.eta(1))^2);
        %a0_360 = rad2deg(wrapTo2Pi(deg2rad(a0)));
        % 
        % 
        % phi_TS = atan2((vessel.eta(2)-tracks.eta(2)), (vessel.eta(1) - tracks.eta(1)));
        % psi_TSR = tracks.eta(3) - vessel.eta(3) - phi_TS;
        % 
        % phi_TS = wrapTo2Pi(phi_TS);
        % psi_TSR = wrapTo2Pi(psi_TSR);
        
        % 1 = HO
        % 2 = GW
        % 3 = SO
        % 4 = OT
        % 5 = SF
        if cflag == 0 %%
            if abs(b0_180) < phi_1 % TS is direcly ahead of OS
                if abs(a0) < phi_1 % TS is facing OS
                    flag = 1;
                elseif a0 > phi_1 && a0 < phi_2 % TS is facing towards OS's starboard
                    flag = 3;
                elseif a0 < (-phi_1) && a0 > (-phi_2) % TS is facing towards OS's port
                    flag = 2;
                else                            % TS is facing away from OS
                    flag = 4; 
                end
            elseif b0 > phi_1 && b0 < phi_2 %TS is ahead on OS's starboard
                if abs(a0) < phi_1
                    flag = 2;   
                elseif a0 > phi_1 && a0 < phi_2
                    flag = 5;
                elseif a0 < (-phi_1) && a0 > (-phi_2)
                    flag = 2;
                else
                    flag = 4;
                end
            elseif b0_180 < -phi_1 && b0_180 > -phi_2 %TS is ahead on OS's port side
                if abs(a0) < phi_1
                    flag = 3;   
                elseif a0 > phi_1 && a0 < phi_2
                    flag = 3;
                elseif a0 < (-phi_1) && a0 > (-phi_2)
                    flag = 5;
                else
                    flag = 4;
                end
            else
                if abs(a0) < phi_1
                    flag = 3;   
                elseif a0 > phi_1 && a0 < phi_2
                    flag = 3;
                elseif a0 < (-phi_1) && a0 > (-phi_2)
                    flag = 3;
                else
                    flag = 5;
                end    
            end
        else % hackjob, needs more work to clear situations properly.
        flag = cflag;
        end
            
        
        %% Woerner method
        % if b0 > 112.5 && b0 < 247.5 && abs(a0) < a13
    %     flag = 'SO';
    % elseif a0_360 > 112.5 && a0_360 < 247.5 && abs(b0_180) < a13,
    %     flag = 'GW';
    % elseif abs(b0_180) < a14 && abs(a0) < a14
    %     flag = 'HO';
    % elseif b0 > 0 && b0 < 112.5 && a0 > -112.5 && a0 < a15
    %     flag = 'GW';
    % elseif a0_360 > 0 && a0_360 < 112.5 && b0_180 < -112.5 && b0_180 < a15
    %     flag = 'SO';
    %     
    % else
    %     flag = 'SO';
    else
        flag = cflag;
    end
    
    if dCPA > (dCPAgrense+30)
        flag = 0;
    end
    
    
    end

\end{lstlisting}


\clearpage
\begin{lstlisting}
    function [N, h] = DynamicHorizon(vessel, dynamic_obs)
% Calculate an appropriate number of time steps and step length based on
% distance to goal and other vessels.

%Distance to goal:
%dist = sqrt((tracks.eta(2) - vessel.eta(2))^2 + (tracks.eta(1) - vessel.eta(1))^2);
distancetogoal = 0;
for i = size(vessel.wp,2):-1:vessel.current_wp+2
    distbetweenWP = sqrt((vessel.wp(1,i) - vessel.wp(1,i-1))^2 + ((vessel.wp(2,i) - vessel.wp(2,i-1))^2));
    distancetogoal = distancetogoal + distbetweenWP;
end
distancetonextWP = sqrt((vessel.wp(1,vessel.current_wp+1) - vessel.eta(1))^2 + ((vessel.wp(2,vessel.current_wp+1) - vessel.eta(2))^2));
distancetogoal = distancetogoal + distancetonextWP;
if vessel.nu(1) < 0.001
    vessel.nu(1) = 0.001;
end
Timetogoal = distancetogoal / vessel.nu(1);


%Getting past relevant TS:
%some function
%return TimetopassTS
if(~isempty(dynamic_obs))
    allTcpas = [dynamic_obs.tcpa];
    maxtCPA = max(allTcpas) + 20; % Add time, we want to pass the encounter, not just reach it.
end

%compare time to pass goal and time to pass TS, we want to keep the
%smallest of theese two

%max time of n minutes:
maxminutes = 5; 
maxseconds = maxminutes * 60;
minminutes = 3;
minseconds = minminutes * 60;

% WRONG
% minstetid = max(minseconds, maxtCPA);
% finaltime = min([Timetogoal, maxseconds, minstetid]);

% CORRECT, but never used
if (false) % <- TODO: check if any cflags are set.
    maxtime = min([Timetogoal, maxseconds]);
    finaltime = min([maxtime, maxtCPA]); 
else
    finaltime = min([Timetogoal, maxseconds]);
end
% finaltime = min([Timetogoal, maxseconds]);




h = 0.5; % statisk for nå.
N = ceil(finaltime / h);

%% HARDCODING
% h = 0.5;
% N = ceil(45 / h);

end
\end{lstlisting}

\clearpage
\begin{lstlisting}
    function feasibility = feasibility_check(previous_w_opt)
    north_opt = previous_w_opt(1:9:end);
    east_opt = previous_w_opt(2:9:end);

    feasibility = 1;

    for i = 1:length(north_opt)-1
        x1 = east_opt(i);
        x2 = east_opt(i+1);
        y1 = north_opt(i);
        y2 = north_opt(i+1);
        dist = sqrt((x2-x1)^2 + (y2-y1)^2);
        if dist > 5
            feasibility = 0;
        end
        %if distance to next point > 5
            % BIG ERROR, NOT FEASIBLE
        %else
            %feasible.

    end
end 
\end{lstlisting}

\clearpage
\begin{lstlisting}
    function [dCPAlist, tCPAlist, pos_OS_list, pos_TS_list] = getCPAlist(vessel,tracks)
dCPAlist = [];
tCPAlist = [];
pos_TS_list = [];
pos_OS_list = [];
wptstimer = 0; % timer used to calculate the position of the other ship at certain wpts.

%%

for i = vessel.current_wp:size(vessel.wp,2)-1
            % NAIV APPROACH
% % % % % % % %     %For each OS transit waypoint, check the dCPA and tCPA for each TS trasit
% % % % % % % %     %waypoint.
% % % % % % % %     [pos_OS, vel_OS] = VesselReadout(vessel,i);
% % % % % % % %     for j = tracks.current_wp:size(tracks.wp,2)-1
% % % % % % % %         [pos_TS, vel_TS] = VesselReadout(tracks,j);
% % % % % % % %         [dCPA, tCPA] = ClosestApproach(pos_OS, pos_TS, vel_OS, vel_TS);
% % % % % % % %         dCPAlist(i,j) = dCPA;
% % % % % % % %         tCPAlist(i,j) = tCPA;
% % % % % % % %     end
            % NAIV APPROACH ^

%Fra vessel.eta, hvor lang tid tar det å nå neste wpt?
%Fra neste wpt, hvor lang tid tar det å nå neste wpt? <- repeat for alle
%wpts. Anta konstant fart hele veien.

    %Find pose at current OS waypoint
    [pos_OS, vel_OS] = VesselWPReadout(vessel,i);
    [pos_TS, vel_TS] = whereisTS(tracks,wptstimer); 

    heading_OS = atan2(vel_OS(2),vel_OS(1));
    heading_TS = atan2(vel_TS(2),vel_TS(1));

    %Do cpa check
    [dCPA, tCPA] = ClosestApproach(pos_OS, pos_TS, vel_OS, vel_TS);
    dCPA = round(dCPA*1000)/1000;
    tCPA = tCPA + wptstimer; % håper dette blir rett.
    pos_OS = [pos_OS;heading_OS];
    pos_TS = [pos_TS;heading_TS];
    pos_OS_list = [pos_OS_list, pos_OS];
    pos_TS_list = [pos_TS_list, pos_TS];


    %step forward one waypoint.
    distancetonextWP = sqrt((vessel.wp(1,i+1) - pos_OS(1))^2 + ((vessel.wp(2,i+1) - pos_OS(2))^2));
    timetonextWP = distancetonextWP / norm(vessel.nu(1:2),2); %Distanse OG time to next wp er redundant, egentlig kunne jeg klart meg med en.
    wptstimer = wptstimer + timetonextWP;
    dCPAlist = [dCPAlist, dCPA];
    tCPAlist = [tCPAlist, tCPA];

  
end
end
\end{lstlisting}

\clearpage
\begin{lstlisting}
    function c_orig = place_dyn_constraint(dynamic_obs,k, i, rad_offset, offsetdist)
    offsetang = atan2(dynamic_obs(i).traj(4,k+1),dynamic_obs(i).traj(3,k+1)) + rad_offset;
    offsetdir = [cos(offsetang);sin(offsetang)];
%     offsetdist = 10; % Should ideally be based some function of Involved vessel's speeds
    offsetvektor = offsetdist*offsetdir;
    c_orig = dynamic_obs(i).traj(1:2,k+1) + offsetvektor;
end
\end{lstlisting}

\clearpage
\begin{lstlisting}
    function static_obs_constraints = Static_obstacles_check_Iterative(obsmatrix, trajectory, k)
%First check if we're using reference LOS trajectory or previous w_opt
    w_opt = 0;
    if size(trajectory,1) > 4
        w_opt = 1;
    end

    %% Initialize
%     startpos = trajectory(1:2,2);
%     heading = vessel.eta(3);
%     heading = atan2(trajectory(2,2)-vessel.eta(2),trajectory(1,2)-vessel.eta(1));
    x = [];
    y = [];
    %% Polygons
    xbox = obsmatrix(2,:);
    ybox = obsmatrix(1,:);

    %Find position
    if(w_opt)
        ypos = trajectory(1:9:end);
        xpos = trajectory(2:9:end);
        if k < length(xpos)
            pos = [ypos(k+1);xpos(k+1)];
        else
            pos = [ypos(length(ypos));xpos(length(xpos))]; % <- Contingency that should never occur.
        end
    else
        pos = trajectory(1:2,k+1);        
    end
    
    %Generate intersection scan lines
    for j = -pi:pi/6:pi
            ang = j; % should probably include heading
            dir = [cos(ang);sin(ang)];
            %% RADIUS OF SCAN HERE
            dist = 50;
            %%
            vektor = dist*dir;
            checkpos = pos + vektor;
            x = [x, pos(2), checkpos(2), NaN];
            y = [y, pos(1), checkpos(1), NaN];
    end

    [xi, yi, ii] = polyxpoly(x,y,xbox,ybox);
    % Keep first hit:
    A = [xi, yi, ii];
    [~,uidx] = unique(A(:,3),'stable');
    A_without_dup = A(uidx,:);
    xi = A_without_dup(:,1);
    yi = A_without_dup(:,2);
    ii = A_without_dup(:,3:4);
    
    %% TEST CODE
%     testx = [];
%     testy = [];
%     mapshow(xbox,ybox,'DisplayType','polygon','LineStyle','none')
%     mapshow(x,y,'Marker','+')
%     mapshow(xi,yi,'DisplayType','point','Marker','o')
%     for i = 1:length(xi)
%         testpoint = [yi(i);xi(i)];
%         line = testpoint - pos;
%         line = [-line(2); line(1)];
%         point1 = testpoint + line;
%         point2 = testpoint - line;
%         testx = [testx, point1(2), point2(2), NaN];
%         testy = [testy, point1(1), point2(1), NaN];
%         mapshow(testx,testy,'Marker','x')
%     end
%%
    %% Generate lines:
    static_obs_constraints = zeros(3,length(xi));
    for i = 1:length(xi)
        intersectionpoint = [yi(i); xi(i)];
        %horrible 2am spaghetti:
        line = pos - intersectionpoint; % The vector that takes us from intersection point current position
        transposedline = [-line(2);line(1)]; % Get Orthogonal of said vector.
        tangent = intersectionpoint + transposedline; % create point along orthogonal vector
        
%         pi_p = atan2(tangent(1) - intersectionpoint(1), tangent(2) - intersectionpoint(2)); % THIS COULD BE OPTIMIZED WITH A TABLE, 
        pi_p = atan2(tangent(2) - intersectionpoint(2), tangent(1) - intersectionpoint(1));
        % check line ID -> lookup corresponding angle :)
        static_obs_constraints(:,i) = [intersectionpoint(1); intersectionpoint(2); pi_p];
        
%         %% Debug code
%         testx = [tangent(2), intersectionpoint(2), NaN];
%         testy = [tangent(1), intersectionpoint(1), NaN];
%         mapshow(testx, testy)
    end
end
\end{lstlisting}

\clearpage
\begin{lstlisting}
    function [pos_OS, vel_OS] = VesselWPReadout(vessel,i)
%Reads out the position and velocity of the vessel at each waypoint in it's
%transit. If the index of the wpt we're reading is the same as current wpt
%we instead read out current position and velocity.
    pos_OS = vessel.wp(1:2,i);
    Heading_OS = atan2(vessel.wp(2,i+1)-vessel.wp(2,i),vessel.wp(1,i+1)-vessel.wp(1,i));
    vel_OS = rotZ(Heading_OS)*vessel.nu;
    vel_OS = vel_OS(1:2);
    if i == vessel.current_wp %When we examine the current actie wp; use current location instead.
        pos_OS = vessel.eta(1:2);
        vel_OS = vessel.eta_dot(1:2);
    end
    
end
\end{lstlisting}

\clearpage
\begin{lstlisting}
    function [pos_TS, vel_TS] = whereisTS(tracks,wptstimer)
    pos_TS = tracks.eta(1:2);
    vel_TS = tracks.eta_dot(1:2);
    WPlim = size(tracks.wp,2);

    pos = tracks.eta(1:2);
    distance = wptstimer * norm(tracks.nu(1:2),2);
    WPindex = tracks.current_wp;
    if WPindex < WPlim
        distancetonextWP = sqrt((tracks.wp(1,WPindex+1) - pos(1))^2 + ((tracks.wp(2,WPindex+1) - pos(2))^2));
    else
        distancetonextWP = 0;
    end
    while distance > 0
        if distance > distancetonextWP
            pos = tracks.wp(1:2,WPindex+1);
            distance = distance - distancetonextWP;
            WPindex = WPindex + 1;
            if WPindex < WPlim
                distancetonextWP = sqrt((tracks.wp(1,WPindex+1) - pos(1))^2 + ((tracks.wp(2,WPindex+1) - pos(2))^2));
            else
                distancetonextWP = 0;
                pos_TS = pos;
                vel_TS = [0,0]';
                distance = 0;
            end
        else
            %Beveg oss (distane) langs banen til neste WP
            %sett distance til null.
            direction = tracks.wp(:,WPindex+1) - pos;
            travel = distance / distancetonextWP;
            pos_TS = pos + travel * direction;
            vel_TS = rotZ(atan2(direction(2),direction(1))) * tracks.nu;
            vel_TS = vel_TS(1:2);
            distance = 0;
        end
    end
end




\end{lstlisting}